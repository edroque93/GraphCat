\documentclass{beamer}
\usetheme{metropolis}

% Layout/Format
\usepackage{multirow}
% Math/Algorithms
\usepackage{amsmath}
\usepackage{amsthm}
\usepackage{amssymb}
\usepackage{algorithm2e}
% Graphics
\usepackage{float}
\usepackage{graphicx}
\usepackage{tikz}
% Utility
\usepackage{dirtytalk}
\usepackage{xcolor}
% Language & Symbols
\usepackage[utf8]{inputenc}
\usepackage[T1]{fontenc}
\usepackage[english]{babel}
\usepackage{wasysym}

\title{Graph Drawing}
\date{\today}
\author{Calvin Bulla \\ Enrique Díaz Roque}
\institute{Algorithms for VSLI}

\begin{filecontents}{refs.bib}
@inproceedings{walshaw2000multilevel,
  title={A multilevel algorithm for force-directed graph drawing},
  author={Walshaw, Chris},
  booktitle={International Symposium on Graph Drawing},
  pages={171--182},
  year={2000},
  organization={Springer}
}
\end{filecontents}

\begin{document}
\maketitle

\begin{frame}
\begin {enumerate}
\item Where (class topic) ~ Global and detailed placement ~ Force directed placement (class theory)
\item Our setup
\item Graph drawing
\item Initial position
\item Iterative process
\item Forces
\item Repulsive
\item Spring
\item Parallelism
\item Experiments modifying functions (forces)
\item Experiments scaling topology
\item Experiments convergence
\item Extensions (clustering, optimizations, details that we didn’t have time to implement (but we liked), …)
\item Conclusion
\end{enumerate}
\end{frame}

\begin{frame}{Initial positioning}
\centering
\only<1>{
$
A=
 \begin{pmatrix}
  0 & 1 & 1 & 0 \\
  1 & 0 & 0 & 1 \\
  1 & 0 & 0 & 1 \\
  0 & 1 & 1 & 0 \\
 \end{pmatrix}
$
}
\only<2>{
$
D=
 \begin{pmatrix}
  2 & 0 & 0 & 0 \\
  0 & 2 & 0 & 0 \\
  0 & 0 & 2 & 0 \\
  0 & 0 & 0 & 2 \\
 \end{pmatrix}
$
}
\only<3>{
Unnormalized Laplacian matrix associated with A\vspace{2em}
$
L=D-A=
 \begin{pmatrix}
  2 & -1 & -1 & 0 \\
  -1 & 2 & 0 & -1 \\
  -1 & 0 & 2 & -1 \\
  0 & -1 & -1 & 2 \\
 \end{pmatrix}
$
}

\end{frame}

\nocite{*}

\begin{frame}{References}
  \bibliographystyle{amsalpha}
  \bibliography{refs.bib}
\end{frame}

\end{document}
